
% Definicja klasy dokumentu
\documentclass[12pt,a4paper]{article}




%------------------------------------
% 			Pakiety
\usepackage[utf8x]{inputenc}
\usepackage{ucs}
\usepackage[MeX]{polski}
\usepackage{fancyhdr}
\usepackage{amsmath}
\usepackage{amsfonts}
\usepackage{amssymb}
\usepackage[hidelinks]{hyperref}
\usepackage{graphicx}
\pagestyle{fancy}

%---------------------------------------
% 		Dodanie strony tytulowej do dokumentu

\renewcommand{\maketitle}{\begin{titlepage}

\begin{center}
%\vspace*{0.7cm}
\large \textsc{POLITECHNIKA WROCŁAWSKA}\\
\large \textsc{WYDZIAŁ ELEKTRONIKI}\\
\vspace*{4.7cm}
\huge \textsc{Sterowniki Robotów}\\
\vspace*{0.7cm}
\LARGE \textsc{Robot mobilny klasy linefollower}
\vspace*{0.7cm}

\vspace*{3cm}


\large

\textsc{Bartosz Lenartowicz} \\
\textsc{Magdalena Kaczorowska} \\



\vspace*{6cm}
\textsc{Wrocław, \today}\\
\end{center}
\end{titlepage}
\newpage
} 

%-----------------------------------------
% 		Poczatek dokumentu
\begin{document}
\maketitle % Komenda maktitle dodaje stronę tytułową do dokumentu głównego
\tableofcontents % Komenda tableofcontents dodaje spis treści do dokumentu głównego
\newpage % Komenda newpage przechodzi do nowej strony w dokumencie.


%				Dokument wlasciwy
\section{Wstęp}

Celem projektu jest wykonanie robota mobilnego klasy linefollower. Zadaniem tego robota jest poruszanie się po czarnej linii na białym tle. Robot ma mieścić się na kartce A4. 

Robot będzie posiadał dwa enkodery AS5048A i komunikował się z nimi za pomocą interfejsu SPI.

\section{Założenia projektowe}
\begin{itemize}

\item Robot będzie składał się z 2 części - płytki głównej, na której będzie znajdował się mikrokontroler wraz z resztą elektroniki i akumulatorem oraz z płytki, na której będą czujniki odbiciowe. 

\item Napęd robota będą stanowiły 2 silniki Pololu.

\item Robot będzie zasilany przez akumulator Li-Pol o napięciu 7,4V i pojemności 300mAh.

\item Zabezpieczenie przed odwrotną polaryzacją w postaci tranzystora MOSFET oraz zabezpieczenie prądowe.

\item Pomiar napięcia baterii - sygnalizacja rozładowania się w postaci diody.

\item Projekt oparty o mikrokontroler STM32F051C8Tx, ponieważ jest mały, a posiada wystarczającą ilość potrzebnych peryferii.

\item 8 czujników odbiciowych KTIR0711s do wykrywania czarnej linii.

\item Silniki sterowane poprzez mostek TB6612.

\item Komunikacja z robotem będzie odbywała się przez moduł Bluetooth. 

\item 2 enkodery AS5048A obsługiwane za pomocą SPI.


Gdy uda się zrealizować powyższe założenia, w planie jest opcjonalne rozszerzenie projektu o dodanie:


\item turbiny ssącej

\item czujników odległośći Sharp


\end{itemize}

\section{Plan pracy}

\subsection{Harmonogram}

\begin{itemize}

\item Zaprojektowanie schematu - 16.04.2017

\item Zaprojektowanie płytki PCB - 7.05.2017

\item Wytrawiona i polutowana płytka - 14.05.2017

\item Wstępne zaprogramowanie robota - 4.06.2017

\item Testy gotowego programu - 18.06.2017

\item Raport końcowy - 22.06.2017

\end{itemize}

\subsection{Kamienie milowe}

\begin{itemize}
\item Zaprojektowana schematu - 16.04.2017

\item Rozpoczęcie programowania robota - 15.05.2017

\item Gotowy projekt wraz z raportem - 22.06.2017
\end{itemize}




\bibliographystyle{plunsrt}
\bibliography{mybib}
\clearpage
\end{document}

%Wzór numerowania

%\section{Punkt}
%\subsection{Podpunkt}


%Wzór dodawania obrazu

%\includegraphics[width=1\textwidth]{lipol.png}
%\caption{Schemat akumulatora \label{fig:lipol}}
%\end{figure}
%\begin{figure}[tp]
%\centering
%\includegraphics[width=1\textwidth]{stabilizator.png}
%\caption{Schemat stabilizatorów \label{fig:stabilizator}}
%\end{figure}
